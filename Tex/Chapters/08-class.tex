\chapter{Classes and Objects}

A C++ class is an user-defined data type that may contain it's own unique methods, attributes, etc. 
To define your own class, use the \verb!class! keyword along with a useful name for it. 

We will be defining our own class, and use it to describe the next sections: 

\begin{verbatim}
#include<strings>
class Summon{
    // Class attributes
    std::string name; 
    char type; 
    int tier; 
    std::string description; 

    // Constructor 
    Summon(std::string name, char type, int tier, std::string description); 

    //Private Methods:
    private:
    int what_dmg(Summon s){ //Finds dmg of a perticular summon
        if (s.type == 'T'){
            return (s.tier * 2 )-1; 
        } else (if s.type == 'S') {
            return s.tier * 2;
        } else {
            return 0;
        }
    }
    std::string what_type(Summon s){
        if (s.type == 'T'){
            return "Tech"; 
        } else if (s.type == 'S'){
            return "Striker";
        }

    }
    //Public Methods: 
    public: 
    void info(Summon s){
        std::cout  << "Name: "<< s.name << "\n" 
        << "Type: " << what_type(s) << "\n" 
        << "Dmg: " << what_dmg(s) << "\n" 
        << "Description: " << s.description << std::endl;
    }
};

\end{verbatim}

\section{Class Members}
A class is comprised of class members:
\begin{itemize}
    \item Attributes, also known as member data, consist of information about an instance of the class.
    \item Methods, also known as member functions, are functions that can be used with an instance of the class.
\end{itemize}


This can be seen in our program above, things like \verb!char type;! or \verb!int tier;! are all class attributes. 
Our class methods were split into \verb!private! and \verb!public! methods *(talked in Access Control below)*, and are 
such things like \verb!what_dmg! or \verb!info!. 

\subsection{Objects}
In C++, an object is an instance of a class that encapsulates data and functionality pertaining to that data.
This can be in our functions like \verb!what_type(Summon s)! that uses the object \verb!Summon s!. 

\section{Constructor}
For a C++ class, a constructor is a special kind of method that enables control regarding how the objects of a class should be created. 
Different class constructors can be specified for the same class, but each constructor signature must be unique.

We have a constructor in our class, and is seen as \verb!Summon(std::string name, char type,! 
\verb!-int tier, std::string description)!, and 
what it means is that to initialize a Summon instance, you must have the following. 

\section{Access Control} 
C++ classes have access control operators that designate the scope of class members:

\begin{itemize}
    \item \verb!public!
    \item \verb!private!  
\end{itemize}


\verb!public! members are accessible everywhere; \verb!private! members can only be accessed from within the same instance of the class or from friends classes.

You can see this when we don't want people to use our \verb!what_dmg! or \verb!what_type! functions, so we can avoid that 
by making those private, while info will be public, and use these private functions. 