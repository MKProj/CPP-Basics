\chapter{Conditionals \& Logic}
\section{if Statement}
An \verb!if! statement is used to test an expression for truth.

If the condition evaluates to true, then the code within the block is executed;  
otherwise, it will be skipped.

\begin{verbatim}
if (a == 10) {
  // Code goes here
}    
\end{verbatim}

\section{else Clause}
An \verb!else! clause can be added to an if statement.

\begin{itemize}
    \item If the condition evaluates to true, code in the if part is executed.
    \item If the condition evaluates to false, code in the else part is executed.
\end{itemize}

\begin{verbatim}
if (year == 1991) {
  // This runs if it is true
}
else {
  // This runs if it is false
}    
\end{verbatim}

\section{else if Statement}
One or more \verb!else if! statements can be added in between the if and else to provide additional condition(s) to check.

\begin{verbatim}
if (apple > 8) {
  // Some code here
}
else if (apple > 6) {
  // Some code here
}
else {
  // Some code here
}    
\end{verbatim}

\section{switch Statement}
A \verb!switch! statement provides a means of checking an expression against various cases.  

\begin{itemize}
    \item If there is a match, the code within starts to execute. 
    \item The \verb!break! keyword can be used to terminate a case.
    \item \verb!default! is executed when no case matches.
\end{itemize}

\begin{verbatim}
switch (grade) {
  case 9:
    std::cout << "Freshman\n";
    break;
  case 10:
    std::cout << "Sophomore\n";
    break;
  case 11:
    std::cout << "Junior\n";
    break;
  case 12:
    std::cout << "Senior\n";
    break;
  default:
    std::cout << "Invalid\n";
    break;
}    
\end{verbatim}

\section{Relational Operators}
Relational operators are used to compare two values and return true or
false depending on the comparison:

\begin{itemize}
    \item \verb!==! equal to
    \item \verb!!=! not equal to
    \item \verb!>!greater than
    \item \verb!<! less than
    \item \verb!>=! greater than or equal to
    \item \verb!<=! less than or equal to
\end{itemize}

\section{Logical Operators}
Logical operators can be used to combine two different conditions.

\begin{itemize}
    \item \verb!&&! requires both to be true (and)
    \item \verb!||! requires either to be true (or)
    \item \verb!!! negates the result (not)
\end{itemize}



