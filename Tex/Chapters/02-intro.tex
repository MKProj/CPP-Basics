\chapter{Introduction}
\section{Program Structure}
Consider the following hello world program, \verb!hello.cpp!: 

\begin{verbatim}
#include<iostream> 

int main(){
    std::cout << "Hello World" << std::endl; 
    return 0;
}
\end{verbatim}

\noindent The program runs from top to bottom, line by line:

\begin{itemize}
    \item The first line instructs the compiler to 
    locate the file that contains a library called \verb!iostream!.
    \item This library contains code that allows for I/O (input \& output).
    \item The \verb!main()! function houses all the instructions for the program.
\end{itemize}

\section{Basic Output}
Now let's talk more about the \verb!std::cout! in our program above. 
This is used to display output to the user's command line or terminal.
To use \verb!std::cout!, you must use it following \verb!<<! and a string
or variable you wish to output. 
\\
\verb!std::cout << "The answer of the test is: " << answer << std::endl;!
\\
\textbf{Note:} \verb!std::endl! is used to end the line of the output. 
\newpage
\section{Comments}
Comments are useful to document code, temporary debugging and in C++, it supports two different 
type of comments, single line \verb!//! and multi-line \verb!/* */!. 
Comments are ignored by the compiler at compiler time, making them a very good way to organize your code. 

\begin{verbatim}
// This single line will be ignored

/* 
The first C++ program written by MKProjects 
was only available for Linux on Sanp!
All of this will be ignored !!!
*/ 
\end{verbatim}

\paragraph{Compile \& Run}
Since we have our program, \verb!hello.cpp!, we may as well compile and run it. 

\begin{verbatim}
# First compile the program with g++ 
$ g++ hello.cpp

# Now run the binary to execute the program 
$ ls
a.out hello.cpp

$ ./a.out 
Hello World
\end{verbatim}