\chapter{Variables}
\par A variable refers to a storage location in the computer’s memory that one can set aside to save, retrieve, and manipulate data.

\par To initialize a variable in C++, you must first declare it's data type (will be discussed next), a name for the variable 
and assigned a value with the assignment operator " \verb!=!"

\noindent Example:  \verb!int var = 10 // data_type name = value; \verb!

\par Variables can also be declared uninitialized, this means it doesn't have a value yet, and this is done by not adding an 
assignment value.

\section{Data Types}
\par In C++, there are 4 primitive data types, these are: 

\begin{enumerate}
    \item integers ( \verb!int!)
    \item double floating point ( \verb!double!)
    \item characters ( \verb!char!)
    \item boolean ( \verb!bool!)
\end{enumerate}
As well as a common data type used along the primitive type is strings( \verb!std::string!). 

\subsection{Integers}
\par The integer data type is a non-decimal number that can be positive or negative. An integer variable 
is declared with the \verb!int! keyword, and keep in mind it can not be manipulated along the double data type. 
An integer typically requires 4 bytes of memory space and ranges from $-2^{31}$ to $2^{31}$.  

\noindent Example: \verb!int foo = 89!

\subsection{Doubles}
\par The double data type is a decimal point number requiring 8 bytes of memory, and is declared using 
the  \verb!double! keyword.  

\noindent Example: \verb!double foo = 0.78!

\subsection{Characters}
\par The character data type is a single character that is wrapped within single quotes \verb!' '!. They typically 
require 1 byte of memory, and is declared by the \verb!char! keyword. 

\noindent Example: \verb!char letter = 'A'!

\subsection{Strings}
\par Strings are an array of characters and are wrapped within double quotes \verb!" "!. To declare a string 
you will need to use \verb!std::string!.

\noindent Example: \verb!std::string word = "hello";!

\subsection{Boolean}
\par A boolean data type is a value that is either \verb!true! or \verb!false!, and is declared using the \verb!bool! keyword 

\noindent Example: \verb!bool condition = true!

\section{Arithmetic Operators}
\par C++ supports different types of arithmetic operators that can perform common mathematical operations:

\begin{itemize}
    \item \verb!+! addition
    \item \verb!-! subtraction
    \item \verb!*! multiplication
    \item \verb!/! division
    \item \verb!%! modulo (yields the remainder)
\end{itemize}

\section{User Input}
\verb!std::cin! which stands for “character input”, reads user input from the keyboard. Here, the user can enter 
a number, press \verb!enter!, and that number will get stored in \verb!tip!.
\begin{verbatim}
int tip = 0;
 
std::cout << "Enter amount: ";
std::cin >> tip;
\end{verbatim}

